\documentclass{article}
\usepackage[b5paper]{geometry}
\usepackage[utf8]{inputenc}
\usepackage[T1]{fontenc}
\usepackage{mathpazo,tgpagella}
\usepackage{microtype}
\linespread{1.05}
\LoadMicrotypeFile{ppl}

\SetProtrusion
   [ %name     = T1-ppl,      % the name is optional
     load     = ppl-T1 ] % 
   { encoding = T1,
     family   = ppl }        % 
   {
     \textendash = {-50, }, \textemdash = {-25, }  % cancel out left protrusion
   }

\begin{document}
\noindent The dramatic circumstances of the Italian Grand Prix and Michael Schumacher’s retirement will live on for a long time. After his rival was sidelined by a bizarre stewards’ decision, Schumacher won the race and then announced his retirement. But it was an amazing few hours, worthy of a scripted piece of drama. BusinessF1 retraced the moves that led to that startling finish.

On Sunday 10th September 2006 at 3:25pm, precisely the same time as Michael Schumacher passed the chequered flag to win the Italian Grand Prix, the staff of Ferrari’s press supremo, Luca Colajanni, started handing an A4 sheet of paper to journalists outside the team’s motorhome. It was a one-page press release announcing the retirement of the most successful racing driver in history, a driver at the top of his game challenging for the world championship. Colajanni had been given precise orders by Ferrari chairman Luca di Montezemolo about just what he had to do and when he had to do it.

It was strange timing, as Schumacher was about to make the announcement himself in the winner’s press conference after the podium ceremony. Normally press releases are handed out after an announcement has been made, or during it – but rarely before. It takes away the point. As so it turned out when half an hour later Schumacher found himself announcing what everybody already knew.

The Ferrari team’s haste to announce its driver’s retirement was indeed bizarre. Colajanni had wanted to pre-empt the driver’s own announcement as if to make sure there was no turning back.

Montezemolo had exercised a strong presence in the Ferrari garage at Monza Park all weekend. On qualifying day he hovered around the Ferrari motorhome waving away journalists’ enquiries about what was going on. On race-day he had arrived with John Elkann, the most senior member of the Agnelli family working at Fiat, and Sergio Marchionne, the chief executive of Fiat. He also had Piero Ferrari in his party. One observer was mystified at the presence of all these big guns and said: “It was as though Luca wanted reinforcements.” But reinforcements for what? It was soon to become clear. Although everything looked normal in the Ferrari garage and motorhome, underneath the surface a civil war was concluding, in Montezemolo’s favour. It had run all summer, but was finally coming to an end. All that Montezemolo now required was for Jean Todt, the team principal, and Michael Schumacher, the number one driver, to run up the white flag.

In truth no one knew what was about to happen. Schumacher didn’t want to retire, at least not that day. And he thought he still retained enough power to get his way. But Montezemolo had long before given him a deadline of Monza and told him (expressly against Jean Todt’s wishes) that it was either driving alongside Kimi Räikkönen in 2007 – or retirement.

In a previous age no one had dared tell Michael Schumacher what to do. He had been king of Formula One for 12 years and for half of them was easily the sport’s most powerful man, eclipsing even Bernie Ecclestone.

Montezemolo hated this situation and had also come to resent Jean Todt’s role in the Michael Schumacher show. He took the Enzo Ferrari view that drivers were employees who performed at the behest of their employers. Todt on the other hand took a collegiate view; the top people at the team, including Schumacher, were his close friends and far from being his employees.

But there is no doubt that this combination of opposing management styles got the job done. And for that reason each had tolerated the other.

Only once before in the 11 seasons that Schumacher had been a Ferrari driver, in 1999, had Montezemolo insisted on getting his way.

Officially, of course, none of the above occurred. The official line was that Schumacher had simply decided to retire many months before and that Ferrari had signed Räikkönen to take his place, end of story. In fact, Todt suggested anyone who thought any different was “stupid”.

Everyone, then, is stupid.

There was clearly tension between Todt and Montezemolo that weekend in Monza. On Friday and Saturday, there had been an uneasy peace as both men went about their business. Then, on race-day, with less than 15 minutes to the start, Montezemolo broke away from Ferrari on the grid and went up to Räikkönen’s car. He leaned over the cockpit and gave a thumbs-up sign, as if indicating that all was going to plan. It was a strange action to pursue with his team’s close competitor at Ferrari’s home race.

After Schumacher’s race victory, Montezemolo was delirious with joy and, flanked by Elkann and Marchionne, in the full glare of television, he embraced Jean Todt and kissed him. But as Montezemolo kissed him Italian style and threw his arms around his shoulders, Todt quickly turned away. It resembled the scene in ‘The Godfather Part III’ when Michael Corleone embraces his brother Fredo whilst whispering his death sentence.

Then it was Michael Schumacher’s turn. After being pecked by Montezemolo, he too resisted his boss’s celebratory embraces and looked blankly over his shoulder. For Montezemolo, as he embraced the two men he knew the press release signalling his victory was being handed out to journalists.

It was now clear to insiders that Montezemolo had won his internal battle with Todt to turn Räikkönen’s option into a firm contract drive for Ferrari in 2007. And it was clear that Schumacher’s ultimatum of ‘Räikkönen’s or me’ had been ignored.

It was a battle Montezemolo had been determined to win. Six years earlier, to give the team the very best chance of winning, he had wanted to hire Mika Häkkinen as team-mate to Schumacher. But he had been blocked by the twin powers of Schumacher and Todt. This time he was determined to prevail. He wanted Räikkönen, and if that meant Schumacher’s departure, then so be it. And he also made it clear he was not prepared to carry on paying Schumacher his US\$45 million a year in his twilight years. In any case that money was no longer available, it had been allocated to Räikkönen in a deal skilfully negotiated by the driver’s manager David Robertson.

In truth Schumacher was not simply being pushed out of Ferrari, he was not prepared to carry on under the terms that were being offered. So he reluctantly decided to retire. And in any event it was good timing – he was going out at the peak of his powers.

Naturally, in the circumstances, the two press conferences, first for TV and then for the press were sad affairs. Schumacher was very morose. He clearly saw no happiness in retirement. But he played the company line and did not vent any feelings of being pushed out. That was not Schumacher’s way. And the timing of the press release before his own announcement had given him no room for manoeuvre. It was done on the express orders of Montezemolo to ensure that he, and not Schumacher, was setting the agenda.

The sense of despair from Schumacher was obvious. He is the one driver on the grid who genuinely loves Formula One. He lives and breathes it. Whilst some other multiple world champions have rushed into retirement, he seemed set to drive on into his 40s. He was clearly not ready to retire after 16 seasons of racing, nearly double the average career span and equalling the career of Ricardo Patrese.

But at the age of 37, he found, like many others, that as far as Montezemolo was concerned he was past his sell-by date. As Schumacher’s long-time manager, Willi Weber, woefully observed in a passing comment to a journalist at Monza: “Michael found he no longer has the power he thought at Ferrari.” So Schumacher’s retirement was just as controversial as his entry into the sport at the Belgian Grand Prix in first practice on Friday 23rd August 1991.

The countdown for Schumacher’s demise had begun on 25th August 2005 when Räikkönen signed a one-year option which gave Ferrari the right, within a certain time period, to employ him, at a salary of around US\$45 million, for three years from 2007 to 2009 with options to renew beyond that. The option price had never been confirmed but was rumoured around the paddock to be US\$5 million.

Everybody knew that the drivers’ market was headed for a shake-up in 2007. It became clear that the contracts of the three best drivers in the world, Schumacher, Räikkönen and Fernando Alonso were all expiring at the same time – at end of 2006. It was a unique event in Formula One history and meant that all three could be driving at different teams in 2007. In normal circumstances one or two of the top drivers might be out of contract at the same time, but never three. However, in truth nobody expected any of the three to move from their incumbent teams. Schumacher was an absolute fixture at Ferrari and showing no sign of retiring. Alonso was winning everything at Renault so why would he move, especially as Flavio Briatore, the Renault team principal, was his manager? And Räikkönen, despite coming to the end of his contract, had options for the future and really nowhere else to go.

And that was how it looked in the summer of 2005 as Räikkönen’s manager, David Robertson, and McLaren Mercedes team principal, Ron Dennis, sat down to discuss the Finnish driver’s future. It was to be the first of the big driver negotiations for 2007.

As far as Robertson was concerned, it was all going to be pretty straightforward. He couldn’t comprehend Räikkönen leaving. The contract was up but Dennis had options to renew it well into the future. These options all stemmed from the original contract Räikkönen had signed in September 2001. Dennis had paid a small fortune to secure Räikkönen’s services including a rumoured US\$14 million to compensate Peter Sauber. It was a complex contract – two years (2002 and 2003) at a modest salary and then three years (2004-2006) for a much larger retainer culminating in the near US\$45 million he was being paid in 2006. But Räikkönen was far from a free agent at the end of his McLaren contract. By all accounts it was at Dennis’s option to take up another three years if he was willing to pay an escalating salary.

Dennis had security, but at a price. There is no way of telling what that price was but it was likely to mean Räikkönen receiving at least US\$60-US\$70 million a year by 2009. But Dennis, who had been bamboozled into agreeing the high price four years before in 2001, just before the 9/11 terrorist attacks when economic conditions had been very different, did not want to pay, although he still wanted Räikkönen to drive for him.

By all accounts Robertson was somewhat surprised, even if he didn’t show it, when Dennis said he wasn’t taking up the option. Although there is no independent confirmation of this it appears that Dennis believed he could cancel the option, and thereby his commitment, and open negotiations with Robertson at a more sensible retainer. After all Dennis believed, and it certainly looked the case, that Räikkönen had nowhere else to go.

It appears Dennis genuinely believed Robertson would simply agree a lower retainer, probably something nearer US\$35 million. But it proved Dennis did not know the man at all. Robertson is an extremely shrewd individual. Even his critics say he can read the minds of team principals. He is believed to study their psyche in his spare time so that he can deal with them more effectively. In his short career in the paddock he has already negotiated with Frank Williams, Flavio Briatore, Ron Dennis and Jean Todt, and bested all of them.

Anyone who has had negotiations with him of any kind is aware of his skills. As one associate says: “He is the sort of man, and this is not said impolitely, with whom one counts ones fingers after shaking his hand. He probably secretly relishes that reputation.”

It is important to emphasise that at that stage of the 2005 season, in spite of Robertson’s reputation, Dennis thought he held all the cards. Räikkönen was dominating the latter half of the 2005 season and McLaren was the top team. Conversely Ferrari was in the doldrums – why would Räikkönen want to go there even if he could?

And Renault was out of the equation. Everyone thought Alonso was a fixture at Renault. When Dennis let Räikkönen’s option lapse he knew, or at least thought he knew, that he could simply wait for Robertson to accept his offer.

But Robertson sensed something different. He sensed discontent in the McLaren organisation, a sense of drift. He had picked up that Adrian Newey was leaving and that Nick Tombazis might do the same. He also thought most of Ferrari’s problems were tyre related and solvable; he knew that Ross Brawn and Rory Byrne had not suddenly become bad engineers.
But Robertson kept his counsel with Dennis and said he would get back to him.

Robertson considered his options and marched over to the Ferrari motorhome to get the lie of the land. He imagined negotiations with Todt alone would be a waste of time. So he sought to engage Montezemolo and Todt together. Again the wily operator had picked up their differences on his radar and thought he might be able to divide and conquer. He was absolutely correct. Whilst Todt was cool to the idea of hiring Räikkönen, Montezemolo was more than keen. But there were complications. Ferrari already had an option with Valentino Rossi and Todt doubted openly that Schumacher would want Räikkönen alongside him. But Robertson spoke privately to Montezemolo. Soon the two men agreed to sign Räikkönen to an option in Ferrari’s favour for a year, and to pay for the privilege.

But Robertson was not out of the woods. At that point he did not think Ferrari would actually sign Räikkönen. But it was his leverage on Ron Dennis. Robertson made sure by judicious leaks to journalist friends that it got around the paddock about Ferrari’s option. Dennis’s bluff had been publicly called.

And so matters rested, until the end of the season when Dennis heard on the grapevine that Räikkönen had signed for Ferrari. Although it was only an option he guessed immediately what was going on and decided he was not about to be kept on a string for a year whilst Ferrari decided his future.

By then the situation with the third driver in the loop, Fernando Alonso, was becoming clouded as rumours spread that Renault would withdraw from Formula One at the end of 2006. One very highly placed pundit whispered in Dennis’s ear that he had heard this would definitely happen. As sad as that might be for Formula One, Dennis realised it was very good news for him. As the rumour gained currency, whatever its truth, it effectively put Alonso into play.

Dennis made an approach for Alonso. He understood, as did everyone else in the paddock, that at around US\$6 million a year, Alonso was underpaid. Dennis offered Alonso US\$16 million a year. The timing of the move was perfect.

At that point Renault’s prospects for 2007 were at their lowest and McLaren’s, after its storming season, at their highest. McLaren had also just announced it had signed Vodafone as title sponsor for 2007; it had more cash than ever. With all things considered Alonso’s manager Flavio Briatore had no choice but to advise his driver to accept Dennis’s offer. He knew Renault at that moment in time would not match it (although later the situation was to change).

Dennis attached one condition to his offer – he wanted to announce it immediately despite the disruption it would cause to his existing drivers. Close friends say he was driven by a desire to get back at David Robertson and tell the Formula One world how clever he was.

Alonso’s signing was announced to an unsuspecting world just before Christmas 2005. It caused a sensation, mainly revolving around Briatore’s position and the obvious conflict of interest. Briatore took it all in his stride. Interestingly he and Dennis came up with entirely different stories of how Alonso was signed. But by then it didn’t matter. After the ravages inflicted on his bank account by David Robertson, Dennis considered it a good day’s work to get Alonso for just US\$16 million.

But Dennis had seriously piqued his existing drivers and when they heard the news both vowed to leave the team at the end of 2006. They felt they had been double-crossed. Räikkönen’s position for 2007 suddenly looked precarious.

Over at Ferrari, Michael Schumacher was as entrenched as ever and the Italian team had signed an option with Valentino Rossi for 2007, this one at the driver’s behest. If Rossi decided to take up his option there would be no room for Räikkönen. The situation was slightly complicated when Rubens Barrichello read the tea leaves and saw that he also would be out at the end of 2006. Honda was desperate to sign him and he negotiated a release from his contract to take a big money, three-year deal. To replace him the team signed Felipe Massa on a one-year contract as a stop-gap. Schumacher expected that it would be him and Rossi in the cockpit for 2007.

But as 2006 began, Montezemolo realised he didn’t want that. Signing Rossi was Todt and Schumacher’s plan. He wanted Räikkönen, his man, in the car for 2007, and started scheming to get his way.

It may seem ridiculous that Montezemolo had effectively to politic within his own company, but that is the way it was. Todt had made Ferrari his own fiefdom, much to the annoyance of Montezemolo. The two had already clashed earlier this year when Montezemolo wanted to take Marlboro off the car for 2007 and find a non-tobacco sponsor. Todt wanted to stay with an eager Marlboro. Montezemolo tried everything he could to find an alternative and even invited Sir Martin Sorrell, chief executive of WPP Group, the world’s biggest advertising agency group, to visit him in Maranello. Ostensibly he wanted to discuss whether WPP and its network of sponsorship agencies could help with finding a new title sponsor for 2007.

But Todt found out about Sorrell’s visit. And when Sorrell arrived at Maranello, he did not meet with Montezemolo but with the Frenchman. Predictably the discussions went nowhere. Todt told Sorrell he already had a title sponsor for 2007 and asked him why he was there. Sorrell wondered that himself and the visit had effectively been a waste of his time. But as Sorrell was leaving, walking down the corridor on his way to Ferrari’s reception, Montezemolo jumped out of a door in front of him and ushered him into a small adjacent conference room. He asked him what had been discussed with Todt and when Sorrell told him, begged him to find an alternative to Marlboro. It was all over in 10 minutes and Sorrell left Maranello shaking his head at the shenanigans he had witnessed between the two men. Sorrell had no intention of wasting his time trying to find a title sponsor for a team that already had one. Todt had already told Sorrell he had done a deal with his friend Louis Camilleri, the chairman of Altria, the Marlboro parent company. Camilleri had agreed a five-year deal from 2007 to pay US\$200 million a year. It was the biggest sponsorship deal ever in Formula One and an offer the team could not turn down.

Montezemolo was in despair after the Marlboro deal was signed. It made Todt, now seen as a top rainmaker, even more powerful inside the team. In fact Montezemolo had begun to feel like a stranger in his own factory. Continually away on Fiat and Italian business, Montezemolo realised he had made a mistake when he had promoted Todt the year before to head the whole Ferrari car factory. He had expected him to fall flat on his face but instead he rose to the task and Ferrari, which had been in the financial doldrums, began a remarkable recovery under Todt’s stewardship.

Montezemolo felt he had created a monster in Todt that he could no longer control. Although the two men had worked together for more than a decade, they were like chalk and cheese. Behind the rough exterior, Todt is a self-made, cultured man, an art lover with impeccable taste. In 2005 he had teamed up with Hollywood actress, Michelle Yeoh, got engaged to her and was in many ways beginning to outshine Montezemolo himself.

By contrast Montezemolo is a proud aristocrat. A member of the Agnelli family by any other name, he is regarded within the Fiat empire as a marketing wunderkind.

No one in Maranello can understand why the Todt-Montezemolo alliance has lasted so long. One observer said: “It is a mystery, Todt’s not Luca’s sort of person and vice versa.”

It was never part of Montezemolo’s plan to get rid of Todt, he simply wanted to break up the Todt-Brawn-Schumacher alliance that so effectively controlled the team. And it appears that the battleground was drawn over Michael Schumacher, with both men determined to get their way.

But Montezemolo was more determined.

Montezemolo was not overawed by Michael Schumacher as so clearly was Jean Todt. That was shown in 1999 when the two men faced up to each other after Schumacher broke his leg at the British Grand Prix. Even after he had recovered enough to go testing Schumacher announced on Sunday 3rd October that he would not be fit enough to take part in the remaining two races of the year in Malaysia and Japan.

After the accident Eddie Irvine had taken up the running for the world championship title and badly needed the help of a strong team-mate. But the last thing Schumacher appeared to want was his team-mate to win the world championship and he had clearly decided, with Todt’s collaboration, to see the last two races out. Irvine pleaded with Montezemolo to intervene.

What happened next was instructive in the differing relationships Schumacher enjoyed with Todt and Montezemolo. On the afternoon of Tuesday 5th October 1999, Montezemolo rang Schumacher at his home in Switzerland to ask if he would change his mind and drive. But Schumacher’s young daughter Gina-Maria answered the phone and told Montezemolo that her Daddy was “getting out of his football boots”. Montezemolo questioned the little girl more closely and ascertained that she and her brother had been enjoying a rough game of football in the garden with their father. When Schumacher finally came to the phone, Montezemolo asked him if indeed he had been playing football. The German had no choice but to be truthful. Once Montezemolo heard that, he said to him that if he was fit enough to play football he was fit enough to drive in Malaysia and Japan. When Schumacher resisted, Montezemolo reminded him that he was being paid US\$2 million a race and would do as he was told. Schumacher had no choice but to comply and on Friday 8th October the team announced he would indeed be returning for the last two races.

The incident had been a lesson for Montezemolo, who realised that a secret conspiracy existed between Todt and Schumacher.

He had run up against it before when he had wanted to hire Mika Häkkinen to partner Schumacher. Then Todt had told Montezemolo that Schumacher would not have it and would leave. In effect Schumacher was so powerful he could dictate terms and Montezemolo could not risk calling his bluff. But Montezemolo believed Schumacher would have stayed and was left smarting by his rebuttal at the hands of the two men.

So when the chance came to sign Kimi Räïkkönen in the summer of 2005, Montezemolo was determined to grab it. After a poor season when the team had won nothing bar the controversial United States Grand Prix, Montezemolo sensed that Schumacher’s reign was coming to an end. He would be nearly 38 when his last contract ended in 2006.

So when David Robertson came calling, Montezemolo was all ears. Robertson brilliantly played off Montezemolo and Todt against each other. According to sources at Ferrari, Montezemolo didn’t want to get into a situation next year where he was looking for a top-line driver and everyone was signed up. Montezemolo is in instinctive man and, as one person close to Ferrari observes: “He decided to put the bunsen-burner under the situation.”

That person confirms that Montezemolo had been bitterly disappointed when he couldn’t sign Häkkinen and it had always rankled: “The aggravation with Todt has been there the whole time but came to a head at Monza. Luca had wanted to see Häkkinen in the other car. He believes it is 200 per cent about the drivers.”

During the 2005 season Montezemolo decided he didn’t want Valentino Rossi even though he had a firm option to join the team. He persuaded Rossi not to take it up and stay in MotoGP. This decision upset Schumacher who could see what it meant. Rossi had had a programme mapped out to familiarise himself with the car prior to a 2007 debut.

Schumacher said at the time: “We are sad not to see him here. I think he has a very high talent and could have done it in terms of driving.” Ross Brawn, the Ferrari technical director and a strong Todt-Schumacher ally was also upset and said: “We were very impressed with what he was able to do. It would have been very exciting. He was very impressive in all the running we did, otherwise we wouldn’t have taken him so seriously. It would have been a nice challenge to have. It’s a shame.”

Rossi’s announcement fuelled speculation that Ferrari had already decided upon its 2007 driver line-up and that Kimi Räikkönen would be named as Michael Schumacher’s team-mate for next season. But by midsummer it was far from decided and a full-scale battle was going on inside Ferrari. There was a stand-off, which would continue until the deadline to take up Räikkönen’s option.

Meanwhile, David Robertson was sensing that Ferrari might not take up Räikkönen’s option and that Schumacher would not drive alongside him. That prompted him to renew relations with Ron Dennis and make sure his options were still open there. But with McLaren’s 2006 car having flopped and the three top technical men, led by Adrian Newey, having left the team, conditions were totally different. So in May, Robertson started serious negotiations with Flavio Briatore to take Räikkönen to Renault. Robertson found a team principal who very badly wanted to do a deal. The downside was that the retainer was half what he had been getting at McLaren and half of what he had been offered at Ferrari. But against that was a very competitive car; in May it was the most competitive car.

The negotiations were a surprise as Briatore had clashed with Robertson in 2001 and openly criticised him and his methods. But now the Italian turned on the charm offensive and entertained Robertson, and his son Steve, on his boat in Monte Carlo. He also introduced them to his ravishing new girlfriend, Elisabetta Gregoracci, and she worked her own charms on the two men as they toured the Renault team principal’s new yacht in Monaco harbour.

Briatore was ready to forget the past if there was a deal to be done. And he badly needed the deal. By this time his position was very different to how it had been in December 2005. Now the future was clear and Carlos Ghosn, the Renault chairman, had made a long-term commitment to the team and even turned on the cash spigot. Briatore was able to offer Räikkönen a decent retainer, said to be US\$21 million but with the added opportunity to accept outside endorsements, which could have been worth another US\$10 million.

The two men held detailed negotiations and Briatore personally spent a lot of time wooing Robertson. Later Briatore would angrily tell friends that he felt Robertson had been wasting his time and had been committed to Ferrari at the same time as he was offering Räikkönen to Renault. However, this was not the case. Robertson had been negotiating in the genuine belief that Ferrari would not take up its option because of Schumacher.

All through the early summer, civil war raged behind the scenes at Maranello. But Schumacher found his power to get his way had gone. Montezemolo appeared not to care whether he stayed or went. At the German Grand Prix, which Schumacher won with Massa second and Räikkönen third, the Ferrari number one driver put on a very public show of affection for his team-mate and totally ignored Räikkönen. It was a classic Schumacher display: he was demonstrating publicly to Montezemolo how he wanted it to be and how good it could be. But Montezemolo was totally unmoved. In fact insiders say it hardened his resolve to dislodge the superstar. And in August, Robertson was proved wrong when Montezemolo signed the contract with him. No one close to Ferrari was surprised, as one insider says: “Luca, being the politician that he is, closed off every rat hole.”

When Schumacher learned the news, he told Montezemolo he wanted until the end of the season to make up his mind about whether he would stay and partner Räikkönen. In the meantime, he didn’t want any announcement made about Räikkönen. But Montezemolo was not having any of that. He wanted the situation resolved and told Schumacher he wanted his decision by Monza, when he would announce Räikkönen. By then it appears Ross Brawn had also decided he would leave if Schumacher did. That news was leaked to journalists to pile pressure on Montezemolo.

The writing was on the wall. Montezemolo had come this far and was not about to turn back.

Montezemolo won the battle: Schumacher would not drive with Räikkönen and would instead announce his retirement. But the decision was very much against his will.

He would have rather carried on with Felipe Massa as his team-mate. Now the seven times world champion, still only 37, has to decide what to do next and where life will take him.

Meanwhile, none of the pronouncements so far can be taken for granted. Despite the 17 years since Enzo Ferrari’s death, Ferrari is still a very Machiavellian organisation and Jean Todt, predictably, is seething about losing this public battle with Montezemolo. He knows he will never have the same type of relationship with Räikkönen that he has had with Schumacher. Insiders, however, insist that Todt’s job is safe and that he has too many friends inside Fiat for Montezemolo to contemplate sacking him. And they add that Montezemolo, who is not regarded as malicious, genuinely doesn’t want that and knows Todt is the best man to run Ferrari. One says: “Whatever Luca is, he isn’t stupid.”

But another outside observer says that Todt has been wounded by what has transpired and doesn’t believe the story is concluded, as he says: “Todt is the most malicious person on two legs and he will hold that against Luca.”

\end{document}
